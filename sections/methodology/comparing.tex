It is important to know if the data that the Ubertooth finds can also be seen in the HCI layer. This will confirm that the data that is shown by the Ubertooth is not bogus. Another reason is that it was unknown what the the packets meant. The correlation between the two layers might give us more clarity about the things we were observing. The data that was inspected were the packets that were not NULL or POLL. These packets have a payload and it is suspected to be received on the HCI level. To compare these packets, the Ubertooth and bt\_snoop from the android device have been used. Unfortunately, the packets on the HCI level have a higher level of abstraction and the packet types seen on the HCI layer are not the same as the packets that are sent over the RF layer. This means the correlation needs to be done with other means than looking at the packet types.\pend The correlation is done by looking at the timestamps. Both the HCI layer and the RF layer have timestamps in their messages. The assumption made is that the timestamps that is at which the packet arrives at the Ubertooth is not much different from the timestamp that is seen in HCI. By looking at the timestamps of the packets on the RF layer and assuming they will arrive at the HCI. The packets can then be linked to the packets seen on the HCI layer.
\subsubsection{Comparing data experiment}
This is the experiment that has been done to correlate the data from the Ubertooth with from the HCI:
\begin{itemize}
\item Install a slightly modified version of the example app called HelloActiveLowPowerActivity( the modification sends a message from the watch to android device once every second)
\item Start \verb|ubertooth-rx -l <LAP -u <UAP|. The UAP will also be given, because Ubertooth will not have to find the UAP by itself. 
\item Then start the smartwatch app that sends the messages
\item Both HCI layer and the Physical layer will be logged.
\end{itemize}
The interval between messages is chosen to be one second per message. With this many messages there is a higher chance of finding the rogue packets. Sending it faster than this will make it harder to link the both layers together, because there will be too many packets with the same timestamp.