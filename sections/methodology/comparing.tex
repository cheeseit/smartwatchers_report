The purpose of looking at different layers is to correlate the packets from the RF with the packets from the HCI level. To compare them, the Ubertooth and bt\_snoop from the android device have been used. The information retrieved from both of the sources is then compared on the timestamps. Unfortunately, the packets on the HCI level have a higher level of abstraction and do not correlate to the packets that are sent over the bandwidth. The assumption made is that the timestamps that is at which the packet arrives at the Ubertooth is not much different from the timestamp that is seen in HCI.
\subsection{Comparing data experiment}
The experiment that has been done to correlate the data is as follows:
\begin{itemize}
\item hallo
\end{itemize}
%The two previous experiments have been done in the same way in order to be able to identify and correlate informations between the two layers.
%Indeed, after analysing raw data from the Ubertooth it was quite straightforward to conclude that the data was encrypted and/or encapsulated. 
%As the data sent over the air was not human readable, it was decided to monitor the same traffic two layers up (before encryption/encapsulation). The two results were then correlated to deduce what was done at the LMP.