During the pairing process some critical information are exchanged. Indeed as explained in the literature review master and slave are encrypting data thanks to several parameters common to each other: a key, a salt and a clock. These parameters are exchanged between master and slave during the pairing. This is what an attacker want to know to be able to decrypt packets and understand what is exchanged.
Many papers have discussed the way to gather these information. Indeed in order to retrieve them, an attacker would have first to see the pairing mechanism. That is why it is require to de-authenticate the user by making it believe his device is not working. Naturally the user will then re-pair himself with his device. This is where the attacker is able to see the different parameters exchanged. 
%This paper will explain how to retrieve these informations and explain how to use these information to decrypt a packet.
The pairing process is done in the Smart Connect application. This happens before any interaction with the smartwatch apps. This means that the apps have no control over the pairing process, but it is interesting to see what is send over. 
To do so the pairing process have been looked from the HCI level of the device. At the HCI it is possible to see all the communication between the smartwatch and the device. During this process the LMP parameters are send over line between the two devices. 
%From these parameters we can see that the encryption flag is turned on and that it uses SSP. (I think to put that in analysis in order to describe how the security is implemented)