\subsubsection{Introduction}
Bluetooth is a wireless communication technology created in 1994 by the mobile telecom company Ericsson. Bluetooth was designed for low-power consumption devices such as sensors, mobile phones, etc. Nowadays more and more devices uses Bluetooth moreover thanks to the uprising of Internet of Things (IoT) and small battery reliant devices.

\subsubsection{Bluetooth communication}
Bluetooth operates in the 2.4 GHz frequency band and uses frequency-hopping spread spectrum. This FHSS means that each packets from a Bluetooth communication is transmitted on one of 79 channels that have a bandwidth of 1MHz each. 
Currently the latest version of Bluetooth is 4.1 that is know as Bluetooth low energy. As its name says BT-LE was designed to reduce power consumption while providing the same communication range and speed that were used in the previous versions. The case study in this project is however not using this latest version however it is using Bluetooth 3.0 + HS (high speed), the version 3.0 + HS is providing a data transfer speed of 24 Mbit/s. Bluetooth is also using a adaptive frequency-hopping spectrum (AFH) used to avoid crowded frequency in the channel spectrum.

\subsubsection{Overview of Bluetooth security}
Bluetooth implements confidentiality, authentication and key derivation with custom algorithms based on the SAFER+ block cipher. (how CA is provided)

\subsubsection{Known Attacks on bluetooth}
