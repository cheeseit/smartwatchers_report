\label{subsubsec:ubertooth}
Bluetooth technology is harder to sniff compared to other wireless protocols such as 802.11 that already have solutions for promiscuous packet sniffing. Because of how the way Bluetooth works it is harder to tap the communication between two devices. The devices need to pair before real message are being send. Normally you can only see the discovery messages with are used to find out which Bluetooth devices want to be found. Another problem is that Bluetooth devices can choose to stay hidden. \\
Due to frequency hopping it is hard to keep track of a device and listen to packet exchanged. However with the Ubertooth project it is now possible. To do so 3 parameters need to be known prior to be able to retrieve any information, the Lower Address Part (LAP), the Upper Address Part (UAP) and the clock CLKN that is the upper 26 bits of the CLK27 of the masters clock. The process to find this parameters is explained later on.
\\
Mike Ossman his project started in 2010 under the name of Ubertooth is dedicated to create and build a open source hardware that is able to follow and sniff Bluetooth communication. The Ubertooth was initially made to do just do following of Bluetooth devices. This can also be seen in the way the plug-ins for kismet work and the names of the tools made for the Ubertooth. The tools will be explained in section \ref{subsubsec:ubertooth_tools}. The latest release (Ubertooth 1) is capable of sniffing communication by identifying the right address and clock. The following sections explain how the clock and address are found and implications of finding these.

%this might not need to be here, but I think it should be somewhere
\subsubsection{The Ubertooth Tools}
\label{subsubsec:ubertooth_tools}
The Ubertooth comes with some tools to help you use the Ubertooth. This is done by the host code which is provided by project Ubertooth. 
The main tool we are using is \verb|ubertooth-rx|  this is the tool that we will be using the most. \verb|ubertooth-rx| is the tool that does all the hard stuff. It listens to all the packets send on the bandwidth and displays them. If you are using the default settings it will try to look for all the UAP of the devices. If you home in on a single device by using the option \verb|-l <LAP>| the tool will do a more in depth analysis. It will then use the given LAP and follow it. Then it will try to find UAP and the clock which will be explained in \ref{subsubsec:finding_uap}. When the program has found these two values it will have enough information to decrypt the packet. This packet will then be printed with. This is exactly where what we are looking for. \pend There is also a Kismet plug-in available which also uses \verb|ubertooth-rx|, but displays it in a better way in the Kismet interface. It also makes log files in pcap format which can then be used to analyse in Wireshark. The Kismet plug-in can only be used to discover all the Bluetooth devices in your neighbourhood and find their UAP. After it found the UAP it will not continue by trying to find the clock.

\subsubsection{Finding the UAP}
\label{subsubsec:finding_uap}
% This might also explain the clock searching process.
% But then I do not know why the clock still needs to found when you have the
% UAP
The UAP is the important to do anything interesting with Bluetooth. For instance it used to determine the hopping sequence that is used for the communication between two devices. It is 8 bits of the 48 bits $BD_ADDR$ (Bluetooth Device address). The LAP is transmitted in plain text. This makes it easy to find as it is present in all the packets. The First part is the NAP(Non-significant Adresss Part). This part is ignored because it is not needed for the initial communication. Because the UAP is only 8 bits it can be brute forced. This is not a good way to find it. Brute forcing is detectable and it will only work when the master is a connectible state. \pend
The Ubertooth is made to be a passive sniffer so this method is not used. One of the techniques used to determine the UAP is by using the HEC(Header Error Check). This is used to check if the header is received without errors. With this it is possible to determine what the are the unknown bits which will be the UAP. This header is send with each packet so it makes it easy to decode. The only problem is that it is XOR with a pseudo-random stream. To try and decode this is called "whitening". There are 64 possible streams it will use and this depends on which master clock is used. The lower 6 bits of are used to determine the. This clock is also used to make the hopping pattern. It will then generate 64 candidate UAP with each of the 64 streams. Sometimes the packets will have a CRC that can be used to check if the right stream is used to decode the packet. \pend
Another method is to perform a series of sanity checks on the packet format. The packet type can then be unwhiten. If the packet type is known some information can be derive such as the packet length. These information can confirm or deny a possible match. The problem with this is that false negatives can happen and it can eliminate UAP streams which are valid. This can happen, because the data that is decoded will be of wrong packet type and it will make the wrong conclusion on whether to or throw away this possible clock. This will lead to not finding any UAP for the master and the process will have to be restarted in order to find it. \pend

\subsubsection{Reason Some packets are not seen on the HCI LEVEL}
The packets we see are only  the null and polling packages. We are only looking at channel 0  on which these packets are send. Sometimes some other packets are send over this channel. These packets are not picked up by the master device. We have found that in the connection set up these packet types are disallowed and this is the reason why these packets are not found in the HCI of the master device.
