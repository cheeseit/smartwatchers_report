\subsection{Traffic investigation}
As shown in the appendix \ref{app:ubertooth} the packets seen are the NULL and POLL packages. The Ubertooth is only looking at channel 0 on which these packets are sent. Sometimes, some other packets are sent over this channel and are not picked up by the master device. From the pairing process on the HCI, in the connection set up these packet types are disallowed. This is the reason why these packets are not found in the HCI of the master device. As you can see in appendix \ref{app:pairing}. The connection settings are set only to accept these kinds of packets shown. All the packets that are send over this channel which are not poll or null are disallowed types. \pend
The reason why these packets are sent is unknown. \textbf{I am pretty sure POLL and NULL are pings = keep alive messages}

From the HCI captures, a few conclusions can be drawn. Indeed, during the process, some parameters are negotiated regarding the future communication between the two devices. More specifically, what we are looking for are the LMP parameters. From there, it is true to say that the communication is encrypted, encapsulated and uses SSP. It is even possible to retrieve the link keys used for encryption at this level.

\subsection{The rogue packets}
As already mentioned, there are some packets we cannot place. These packets can be found by the Ubertooth, but cannot be detected by the HCI. One possibility is that these packets are not meant for the master device. Since the packet types are disallowed by the master it is quite possible, but then the question is why are these sent. Another explanation is that these packets are just the same polling and null packets, but they are just badly decrypted. There is no correlation to be found by comparing the different rogue packets. In fact, there may be some correlation, but it is not clear to guess that by just looking at them as a single set. A possible solution is to have a similar tool as the ubertooth on the smartwatch to see what is sent at the same time. \\
As seen in appendix \ref{app:roguepackets}, there is an explanation of what information we get about the packet.
Looking at the first packet in appendix \ref{app:roguepackets} there are a few things that can be seen. The data is always doubled and the packet is divided in a few fields. Type, LT\_ADDR, flow, payload length and data where the first line contains some information about the packet. The most important values from this information are:
\begin{itemize} 
\item The \textbf{ch}, this is the channel in which the packet was found.
\item The \textbf{LAP}, which is the sender's lower address part.
\end{itemize}
%Explain the packet. 