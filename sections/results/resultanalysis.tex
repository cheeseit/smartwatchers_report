\subsection{Traffic investigation}
\subsubsection{RF layer}
\label{subsubsection:rflayer}
Regarding the RF layer a few messages can be seen there. As shown in the appendix \ref{app:ubertooth} the most packets seen are the NULL and POLL messages, that are keep alive messages. During the captures the Ubertooth is only looking at channel 0 on which these packets are sent.

Some other packets are decoded by the Ubertooth and contains some data, the packets over the RF layer are of a different types. The packet decoded are of type DV/3-DH1, AUX1, AFH, DM1, HV1, DH5/3-DH5. An explanation of these packet types are explained in this appendix \ref{app:types} .

The only valuable information that could be output from this data analysis is that the packets sent between the smartwatch and the phone are fragmented as they contain different LLID parameters (which specifies if the payload is the start or the continuation of a L2CAP or LMP message). The packets are also encrypted.

\subsubsection{HCI layer}
From the HCI captures, some conclusions can be drawn. Indeed, during the pairing process, some parameters are negotiated regarding the future communication between the two devices. SSP introduces IO capabilities that permits to exchanged what would be the pairing model based on the capability of the master and slave device. 
\begin{figure}[!h]
  \begin{center}
	\includegraphics[width=270px]{images/IO_PARAM.png}
	\label{fig:io}
	\caption{IO capabilities found during the pairing process at the HCI layer}
  \end{center}
\end{figure}
The pairing process analysing shows that the pairing process the watch and the phone is based on JustWork mechanism. Indeed, as shown in \ref{fig:io} the Smartwatch claim not to have any input or output and that it does not have OOB authentication then the two devices agree to set up their connection with the JustWork mechanism that involves automatic accept of the numeric comparison. This mechanism is not protected from a MiTM attack.

Then the LMP parameters \ref{fig:lmp} are exchanged to choose the LMP parameters that will be used during this communication. 
\begin{figure}[!h]
  \begin{center}
	\includegraphics[width=270px]{images/LMP_PARAM.jpg}
	\label{fig:lmp}
	\caption{LMP Parameters found during the pairing process at the HCI layer}
  \end{center}
\end{figure}

From there, it is true to say that the communication is encrypted, encapsulated and uses SSP. It is even possible to retrieve the link keys used for encryption at this level. The link keys are always updated after a new pairing.

\subsubsection{Correlation of the two layers}

%Why cant we correlate the packet seen on Rf to hci? 
%Maybe we should trey to decrypt a packet using the key we found nat the hci. Just to proove that it is possible to decrypt the packets if we would have the key. ANd we would have the key maybe if the ubertooth was doing what we did.
Firstly the POLL and NULL packet are not available at the HCI layer. They are destined and processed at the link controller layer so they cannot be seen. With this said these package will have no effect on the correlation between what is seen on the Ubertooth and in the HCI. Also POLL and NULL packets are not interesting in general. More interesting are the other packages mentioned in section \ref{subsubsection:rflayer}. These packets can be see every so often and they come in a wide variety and have different payloads.
These are packets that can be found and unwhitened by the Ubertooth.
Some examples are seen in appendix \ref{app:roguepackets}.
Looking at the first packet in appendix \ref{app:roguepackets} there are a few things that can be seen. The packets with data are always doubled. The reason for this is unknown. The unwhitened packet is divided in a few fields. Type, LT\_ADDR, flow, payload length and data. The first line contains some information about the packet. The most important values from this information are:
The \textbf{ch}, this is the channel in which the packet was found.
The \textbf{LAP}, which is the sender's lower address part.
The reason that the type has two values is that for different data rates the packet type can differ. The SW2 has no voice capabilities and knows high data transfer rate. In specifications table that can be found 308 in the last column. So the packet types will always be the latter of the two types given.
The \textbf{LT\_ADDR} is used to determine which is used by each receiving device to determine if the packet addressed to them, but it is also used for internal routing.
The \textbf{LLID} is the (Logical Link Identifier). This is used to give information about the packet. For example if the value is $2$ this means that this is the start of the packet or it is standalone. It can also indicate that it is control information between the link manager of the master and the slave. This is indicated by a LLID of $3$.
The \textbf{flow}
For more detailed explanation of these fields look in the specifications that can be found here \cite{bt3.0}.
After seeing the Bluetooth specifications it is no surprise that these packets cannot be seen on the HCI layer. These packets are used for the control of the data packages and other control information between master and and slave. These packets are not relevant for the higher layers and cannot be seen.
%BUT WE FAILED/ ANALYSED HOW WE CAN DO THAT