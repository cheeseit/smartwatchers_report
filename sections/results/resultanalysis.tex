\subsection{Traffic investigation}
As shown in the appendix \ref{app:ubertooth} the packets seen are the null and polling packages. The Ubertooth is only looking at channel 0 on which these packets are send. Sometimes some other packets are send over this channel. These packets are not picked up by the master device. From the pairing process that can be seen from HCI that in the connection set up these packet types are disallowed. This is the reason why these packets are not found in the HCI of the master device. As you can see in appendix \ref{app:pairing}. The connection settings are set only to accept these kinds of packets shown. All the packets that are send over this channel which are not poll or null are of the types that are disallowed. \pend
The reason why these packets are send is unknown.

From the HCI captures a few conclusions can be draw. Indeed during the process some parameters are negotiated regarding the future communication between the two devices. More especially what we are looking for is the LMP parameters. From them it is true to say that the communication is encrypted, encapsulated and uses SSP. It is even possible to retrieve the link keys used for encryption at this level.

\subsection{The rogue packets}
As already mentioned the packets there are some packets we cannot place. These packets can be found by the Ubertooth, but cannot be detected by the HCI. One possibility that these packets are not meant for the master device. Since the packet types are disallowed by the master it quite possible, but then the question is why are these send. Another explanation is that these packets are just the same polling and null packets, but they are just badly decrypted. There is no correlation to be found by comparing the different rogue packets. Maybe there is some correlation, but it is not clear at by just looking at them as a single set. A possible solution is to have a similar tool on the smartwatch to see what is send. As seen in appendix \ref{app:roguepackets}. Here is an explanation of what information we get about the packet.
Looking at first packet in appendix \ref{app:roguepackets} there a few things that can be seen. The data is always doubled and the packet is divided in a few fields. Type, LT\_ADDR, flow, payload length and data. The first line contains some information about the packet. The most important values from this are:
\begin{itemize} 
\item The \textbf{ch} this is the channel the packet was found 
\item The \textbf{LAP} the device that has send the message
\end{itemize}
%Explain the packet. 