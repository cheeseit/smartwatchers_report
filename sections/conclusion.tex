The Ubertooth is not useful for inspecting the communication between the SW2 and another device. Packets that can be seen with the Ubertooth are the discovery, keep alive and  "rogue" packets. This makes the Ubertooth useless to inspect the data exchange between the SW2 and another device android. The reason for this is that SW2 uses Bluetooth 3.0. With Bluetooth 3.0 most of the packets are send with a data rate that Ubertooth cannot follow. On the other hand we can confirm that the packets that can be picked up by the Ubertooth can be unwhitened although the payload of the packets are still encrypted. \pend 
However some interesting point came up when investigating the HCI layer. First of all the SW is using the JustWork protocol. Which means that no PIN code exchanged/comparison is needed between the two devices. That makes the pairing process weak against MiTM attack. With a MITM a attack all information can be seen and data can be altered. In order to solve this issue Sony should force the use of a stronger SSP association model such as NFC (already available) or the Passkey entry.   

Ubertooth is still under development. With time the tools will get better. Seeing what the Ubertooth can with Bluetooth 4.0 BTLE it is already a nice tool to have. But for Bluetooth 3.0 it is not useful. However a solution could be to use a tool such as the FTS4BT Frontline sniffer \cite{FTS4BT}. This sniffer is able to sniff any Bluetooth protocol. The only issue with this hardware is that it is expensive and is not open source.
\pend
In the future, experiments can be done with a smartwatch that Bluetooth 4.0 Low energy which transfer data in low energy mode. The Ubertooth has good tools to work with this and it is still nice to look what you can do with it.\pend
Further inspecting what the Ubertooth can and cannot detect. One idea for this is to manually setting up a connection between devices and send different kinds of data. During the project all connections have been setup automatically. \pend
The Smart Connect app can be investigated as well. This app controls all the smartwatch apps and plays a pivotal role in how the smartwatch works. Inspecting this might surface more vulnerabilities not only in the smartwatch, but also other Bluetooth devices. \pend
Lastly the MITM attack can be performed. It has already been shown that if your device uses JustWork for pairing that it is vulnerable. It will be interesting to see how the smartwatch apps will react to a MITM. 