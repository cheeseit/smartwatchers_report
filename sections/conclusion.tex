The Ubertooth is not useful for inspecting the communication between the SW2 and another device. Packets that can be seen with the Ubertooth are the discovery, keep alive and  "rogue" packets. This makes the Ubertooth not very useful to do any kind of analysis about the traffic between the SW2 and another device. This is because the SW2 uses Bluetooth 3.0. With Bluetooth 3.0  most of the packets are send with a data rate that Ubertooth cannot follow. On the other hand we can confirm that the Ubertooth can be used follow discoverable Bluetooth devices on the network and can uniquely identify them. \pend 
However some interesting point came up when investigating the HCI layer. First of all the SW is using the JustWork protocol. Which means that no PIN code exchanged/comparison is needed between the two devices. That makes the pairing process weak against MiTM attack. In order to solve this issue Sony should force the use of a stronger SSP association model such as NFC (already available) or the Passkey entry.   

Ubertooth is still under development. The tools are still being improved and there is not enough documentation about how to properly use them. With time this will get better and seeing what it can already do with Bluetooth 4.0 it is nice tool for a variety for experiments. But for Bluetooth 3.0 it is not a very useful tool. However a solution could be to use a tool such as the FTS4BT Frontline sniffer \cite{FTS4BT} that are able to sniff any Bluetooth protocols. The only issue with this hardware is that it is really expensive. Then it was not possible to acquire this kind of hardware for this project due to budget issue.
\pend
In the future experiments can be done with a smartwatch that Bluetooth 4.0 which transfer data in low energy mode.
Further inspect what the Ubertooth can and cannot pick up. One thing that can be done is manually setting up a connection between devices and send different kinds of data. During the project this has been done with some standard apps that are used available on the SW2, but with some more control maybe more can be seen. \pend
The MITM attack can be performed. It has already been shown that if your device uses JustWork for pairing that it is vulnerable. It will be interesting to see how the smartwatch apps will react to a MITM. To make this possible a normal dongle should be bought and a Bluetooth spoofing program.