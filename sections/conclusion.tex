The tools used in this project are not useful for inspecting the communication between the SW2 and another device. However some interesting point came up when investigating the HCI layer. First of all the smartwatch is using the JustWork protocol. Which means that no PIN code exchanged/comparison is needed between the two devices. That makes the pairing process weak against MiTM attack. Things that can be seen are the discovery, keep alive and  "rogue" packets. This makes the Ubertooth not very useful to do any kind of analysis about the traffic between the SW2 and another device. This is because the SW2 uses Bluetooth 3.0. With Bluetooth 3.0  most of the packets are send with a data rate that Ubertooth cannot follow. \pend  On the other hand the Ubertooth can be used to the follow discoverable Bluetooth devices on the network and can uniquely identify them. The SW2 is easy is program. 
%From what we can see from the pairing process was that is the data send is encrypted and that is the key exchange is done properly.
Ubertooth is still under development. The tools are still being improved and there is not enough documentation about how to properly use them. With time this will get better and with the if Bluetooth 4.0 stays the way it is it will be tool that can be nice to use do all kinds of fun stuff. But for Bluetooth 3.0 it is not a very useful tool. However a solution could be to use a proprietary tool such as the FTS4BT Frontline sniffer \cite{FTS4BT} that are able to sniff any Bluetooth protocols. The only issue with this hardware is that it is really expensive. Then it was not possible to acquire this kind of hardware for this project due to budget issue.
\pend
In the future experiments can be done with a smartwatch that Bluetooth 4.0 which transfer data in low energy mode. The current focus of the Ubertooth project is on the Low Energy mode of Bluetooth 4.0, because this is something the Ubertooth can easily decrypt and follow. 
To further inspect what the Ubertooth can and cannot pick up. One thing that can be done is manually setting up a connection between devices and send different kinds of data. During the project this has been done with some standard apps that are used avaiable on the SW2, but with some more control maybe more can be seen. \pend