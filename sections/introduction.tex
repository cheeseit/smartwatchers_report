
With the uprising of IoT (‘Internet of Things’), more and more wearable devices start to get connected. One of these wearable devices is the smartwatch. It is mostly used to display informations or notifications coming from a phone through a Bluetooth communication. It can also be used to collect data. These data exchanged through the air are most of the time private information and could be sensitive. That is why it is more than required to ensure that this data is secure.

With all new technology comes new security concerns. Security issues are even more relevant in the case of small battery reliant devices. As encryption and security can be very computation intensive, this will take huge toll on these devices. The aim of this project will be to investigate how secure the communication between a smartwatch and a smartphone is, and try to find solutions or alternatives to the problems found (if any) in our investigation.

This project has been conducted with an Android phone (version 4.0 or plus) that will act as the master and a Sony SW2 Smartwatch that will act as a slave. This smartwatch is one of the newest and most available on the market. To gather information, a Bluetooth sniffer named Ubertooth One has been used.

This paper will firstly present the Bluetooth technology by introducing how it is working, why is it used and which version are in use nowadays. Related work in the Bluetooth area will then be presented followed by the methodology of our project. An analysis of the results will outline the possibility of hacking a Bluetooth device and finally conclusions and future work will end this paper.