
With the uprising of IoT (‘Internet of Things’), more and more wearable devices start to get connected. One of these wearable devices is the smartwatch. It is mostly used to display informations or notifications coming from a phone through Bluetooth communication. It can also be used to collect data. The data exchanged through the air is most of the time private information and could be sensitive. That is why it is important that this data is secure.\\

With all new technology comes new security concerns. Security issues are even more relevant in the case of small battery reliant devices. As encryption and security can be very computation intensive, this will take huge toll on battery lives of these devices. The aim of this project will be to investigate how secure the communication between a smartwatch and a smartphone is, and try to find solutions or alternatives to the problems found (if any) in our investigation.\\

This project has been conducted with an Android phone (version 4.0 or plus) that will act as the master and a Sony SW2 Smartwatch that will act as a slave. This smartwatch is one of the newest and most available on the market. To gather information, a Bluetooth sniffer named Ubertooth One has been used.\\

This paper will firstly present in the literature review  how the Bluetooth technology works and also introduce the tools used for this research. The methodology followed during this project will then be presented. An analysis of the results will outline the possibility of hacking a Bluetooth device and why we couldn't hack it ourselves. Finally, the conclusion and possible future work will be described.
\newpage
\subsection{Research Question}
Information exchanged between smartwatches and smartphones is private and needs good security in order not to leak any sensitive information. 
The main means of communication between smartwatches and smartphones is Bluetooth.
The use of low-power communication makes it more likely that there will be security issues with the communication and that encryption might not be set up as it is asking a lot of computation resources. 
This paper will try to answer these questions:
\begin{itemize}
\item[•] How secure is the Bluetooth communication between smartphone and smartwatches?
\item[•]Is it possible to eavesdrop any data exchanged during this communication?
\item[•]Is it possible to change the content of the data exchanged?
\item[•]Is it possible to take control of a smartwatch and/or the smartphone?
\item[•]How the vulnerabilities found (if any) could be addressed?
\end{itemize}

\subsection{Ethics}
From an ethical point of view, there will be no major issues for the experimentation that will be conducted. The equipment and the data used will be the authors/university property. The authors will not eavesdrop other people devices that are not involved in the experimentation.
The publication of our paper could bring ethical issues as the smartwatch market is expanding and indeed, if this research comes up with security flaws they could be used in an evil way by others. That is why the paper produced will evaluate the ethical problems brought by the research (if any).